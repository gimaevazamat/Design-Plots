\documentclass[12pt, a4paper]{article}
\usepackage[T2A]{fontenc}
\usepackage[utf8]{inputenc}
\usepackage[russian]{babel}
\usepackage{geometry}
\geometry{a4paper, margin=1.5cm, bottom=2cm}

\usepackage{subcaption}
\usepackage{graphicx}
\usepackage{xcolor,mdframed}
\usepackage{graphicx}
\usepackage{fancyvrb}
\usepackage{color}
%\usepackage{hyperref}

\usepackage{fancyhdr}
\pagestyle{empty}

\usepackage[pdftex,
		pdfauthor={Azamat Gimaev},
		pdftitle={Design-Plots},
		colorlinks=true,
		linkcolor=blue,
		filecolor=magenta,      
		urlcolor=red
	]{hyperref}


\usepackage{etex}
\usepackage{pgfplots}
\pgfplotsset{compat=newest}
\usepackage{pgfplotstable}
\usepackage{tikz}
\usepackage{float}
%\tikzset{/pgf/number format/.cd,1000 sep={,}}




\usepackage{manual}

\usepackage{listings}
\lstnewenvironment{code}{\lstset{
		backgroundcolor=\color{blue!20},
		breaklines=true
}}{}

\def \gitref {https://github.com/gimaevazamat/Design-Plots}

\begin{document}
%\noindent v1.0 Актуальная версия \href{https://drive.google.com/file/d/1V7IAhQuAOamahaL_1-x6jXTo4OvkqJZa/view?usp=drive_link}{здесь}
%\tableofcontents\newpage
\begin{huge}
	\begin{center}
		\textbf{Графики в PGFPlots}\\
		\small \textit{by Azamat Gimaev} \\
		\large v0.1. Актуальная версия \href{\gitref}{здесь}
	\end{center}
\end{huge}
\section{Введение}
Более полный и качественный мануал по PGFPlots в \href{https://texdoc.org/serve/pgfplots/0}{книге}.
\subsection{Необходимые пакеты}
Установите в преамбуле следующие пакеты:
\begin{code}
	\usepackage{pgfplots}
	\pgfplotsset{compat=newest}
	\usepackage{pgfplotstable}
	\usepackage{tikz}
\end{code}
\subsection{Основы принципа построения графиков в PGFPlots}
PGFPlots базируется на основе TikZ, пакета для создания графических изображений в \LaTeX. Поэтому документация для этого пакета может вам понадобится, она находится на \href{https://texdoc.org/serve/pgfplots/0}{сайте}.
\begin{figure}[H]
\begin{subfigure}[c]{0.5\linewidth}
\centering
\begin{tikzpicture}[example]
\begin{axis}
	...
\end{axis}
\end{tikzpicture}
\caption*{Рис. 1: Подпись}
\end{subfigure}
\begin{subfigure}[c]{0.5\linewidth}
\begin{code}
	\begin{figure}
	\begin{tikzpicture}
		\begin{axis}
		...
		\end{axis}
	\end{tikzpicture}
	\caption{Подпись}
	\end{figure}
\end{code}
\end{subfigure}
\end{figure}
\section{Оси}
\subsection{Расположение осей}

Расположение осей задается параметрами \verb|axis x line| и  \verb|axis y line| для горизонтальной и вертикальной осей соответственно. \verb|axis lines| задаст положения для двух осей сразу. \verb|*| в конце сделает ось без стрелки.
\begin{table}[H]
	\centering
	\begin{tabular}{|c|c|}
		\hline
		\verb|axis x line|&\verb*|box|, \verb*|center|, \verb*|middle|, \verb*|none|, \verb*|top|, \verb*|bottom| \\ \hline
		\verb|axis x line*|&\verb*|box|, \verb*|center|, \verb*|middle|, \verb*|none|, \verb*|top|, \verb*|bottom| \\ \hline
		\verb|axis y line|&\verb*|box|, \verb*|center|, \verb*|middle|, \verb*|none|, \verb*|right|, \verb*|left|\\ \hline
		\verb|axis y line*|&\verb*|box|, \verb*|center|, \verb*|middle|, \verb*|none|, \verb*|right|, \verb*|left|\\ \hline
		\verb|axis lines|&\verb*|box|, \verb*|center|, \verb*|middle|, \verb*|none|, \verb*|right|, \verb*|left|\\ \hline
		\verb|axis lines*|&\verb*|box|, \verb*|center|, \verb*|middle|, \verb*|none|, \verb*|right|, \verb*|left|\\ \hline
		  
	\end{tabular}
\end{table}

\begin{figure}[H]
\begin{subfigure}[c]{0.55\linewidth}
\centering
\begin{tikzpicture}[example]
	\begin{axis}[
		axis lines = box,
		ymin=-1, ymax=1,
		xmin=0, xmax=1,
		xlabel=$x$, ylabel=$y$,
		]
		...
	\end{axis}
\end{tikzpicture}
\end{subfigure}
\begin{subfigure}[c]{0.45\linewidth}
\begin{code}
	\begin{tikzpicture}
		\begin{axis}[
			axis lines = box,
			ymin=-1, ymax=1,
			xmin=0, xmax=1,
			xlabel=$x$, ylabel=$y$,
		]
		...
		\end{axis}
	\end{tikzpicture}
\end{code}
\end{subfigure}
\end{figure}

\begin{figure}[H]
	\begin{subfigure}[c]{0.55\linewidth}
		\centering
		\begin{tikzpicture}[example]
			\begin{axis}[
				axis x line* = middle,
				axis y line* = right,
				ymin=-1, ymax=1,
				xmin=0, xmax=1,
				xlabel=$x$, ylabel=$y$
				]
				...
			\end{axis}
		\end{tikzpicture}
	\end{subfigure}
	\begin{subfigure}[c]{0.45\linewidth}
		\begin{code}
	\begin{tikzpicture}
		\begin{axis}[
			axis x line* = middle,
			axis y line* = right,
			ymin=-1, ymax=1,
			xmin=0, xmax=1,
			xlabel=$x$, ylabel=$y$
			]
			...
		\end{axis}
	\end{tikzpicture}
		\end{code}
	\end{subfigure}
\end{figure}

\begin{figure}[H]
	\begin{subfigure}[c]{0.55\linewidth}
		\centering
		\begin{tikzpicture}[example]
			\begin{axis}[
				axis x line = bottom,
				axis y line = left,
				ymin=-1, ymax=1,
				xmin=0, xmax=1,
				xlabel=$x$, ylabel=$y$
				]
				...
			\end{axis}
		\end{tikzpicture}
	\end{subfigure}
	\begin{subfigure}[c]{0.45\linewidth}
		\begin{code}[tabsize=3]
	\begin{tikzpicture}
		\begin{axis}[
			axis x line = bottom,
			axis y line = left,
			ymin=-1, ymax=1,
			xmin=0, xmax=1,
			xlabel=$x$, ylabel=$y$
			]
			...
		\end{axis}
	\end{tikzpicture}
		\end{code}
	\end{subfigure}
\end{figure}
\subsection{Подписи к осям}
По умолчанию расположение \verb|xlabel| и \verb|ylabel| задаются так:
\begin{figure}[H]
	\begin{subfigure}[c]{0.55\linewidth}
		\centering
		\begin{tikzpicture}[example]
			\begin{axis}[
				axis lines = left,
				xlabel=$x$, ylabel=$y$,
				ymin=-1, ymax=1,
				xmin=0, xmax=1,
				label style={at={(ticklabel cs:0.5)},anchor=near ticklabel}
			]
				\addplot[blue] {x};
			\end{axis}
		\end{tikzpicture}
	\end{subfigure}
	\begin{subfigure}[c]{0.45\linewidth}
		\begin{code}
	\begin{tikzpicture}
		\begin{axis}[
				xlabel=$x$, ylabel=$y$,
				label style={
				at={(ticklabel cs:0.5)},
				anchor=near ticklabel}
			]
			\addplot[blue] {x};
		\end{axis}
	\end{tikzpicture}
		\end{code}
	\end{subfigure}
\end{figure}
Другие варианты:
\begin{figure}[H]
	\begin{subfigure}[c]{0.55\linewidth}
		\centering
		\begin{tikzpicture}[example]
			\begin{axis}[
				axis lines = left,
				xlabel=$x$, ylabel=$y$,
				ymin=-1, ymax=1,
				xmin=0, xmax=1,
				xlabel style={at={(ticklabel* cs:1)},anchor=south, yshift=1.5ex},
				ylabel style={at={(ticklabel cs:0.2)},rotate=-90,anchor=near ticklabel}
				]
				\addplot[blue] {x};
			\end{axis}
		\end{tikzpicture}
	\end{subfigure}
	\begin{subfigure}[c]{0.45\linewidth}
		\begin{code}
	\begin{tikzpicture}
		\begin{axis}[
			xlabel=$x$, ylabel=$y$,
			xlabel style={
				at={(ticklabel* cs:1)},
				anchor=south, 
				yshift=1.5ex},
			ylabel style={
				at={(ticklabel cs:0.3)},
				rotate=-90,
				anchor=near ticklabel}
			]
			\addplot[blue] {x};
		\end{axis}
	\end{tikzpicture}
		\end{code}
	\end{subfigure}
\end{figure}
\begin{figure}[H]
	\begin{subfigure}[c]{0.55\linewidth}
		\centering
		\begin{tikzpicture}[example]
			\begin{axis}[
				axis lines = left,
				xlabel=$x$, ylabel=$y$,
				ymin=-1, ymax=1,
				xmin=0, xmax=1,
				xlabel style={at={(0.5, 0.5)}, scale=1.5},
				ylabel style={at={(0.1, 0.8)},rotate=-90, color=red}
				]
				\addplot[blue] {x};
			\end{axis}
		\end{tikzpicture}
	\end{subfigure}
	\begin{subfigure}[c]{0.45\linewidth}
		\begin{code}
	\begin{tikzpicture}
		\begin{axis}[
			xlabel=$x$, ylabel=$y$,
			xlabel style={
				at={(0.5, 0.5)},
				scale=1.5},
			ylabel style={
				at={(0.1, 0.8)},
				rotate=-90,
				color=red}
			]
			\addplot[blue] {x};
		\end{axis}
	\end{tikzpicture}
		\end{code}
	\end{subfigure}
\end{figure}

%Подробнее про систему координат и подписи к осям можно найти на \href{https://texdoc.org/serve/pgfplots/0}{стр. 189}.
\subsection{Метки на осях}
Настройка расстояний между тиками:
\begin{figure}[H]
	\begin{subfigure}[c]{0.55\linewidth}
		\centering
		\begin{tikzpicture}[example]
			\begin{axis}[
				axis lines = box,
				xlabel=$x$, ylabel=$y$,
				ymin=-1, ymax=1,
				xmin=0, xmax=1,
				xtick distance=0.1, ytick distance=0.25
				]
				\addplot[blue] {x};
			\end{axis}
		\end{tikzpicture}
	\end{subfigure}
	\begin{subfigure}[c]{0.45\linewidth}
		\begin{code}
	\begin{tikzpicture}
		\begin{axis}[
			axis lines = box,
			xlabel=$x$, ylabel=$y$,
			ymin=-1, ymax=1,
			xmin=0, xmax=1,
			xtick distance=0.1,
			ytick distance=0.25
			]
			\addplot[blue] {x};
		\end{axis}
	\end{tikzpicture}
		\end{code}
	\end{subfigure}
\end{figure}
Также можно вручную указать список основных тиков:
\begin{figure}[H]
	\begin{subfigure}[c]{0.55\linewidth}
		\centering
		\begin{tikzpicture}[example]
			\begin{axis}[
				axis lines = box,
				xlabel=$x$, ylabel=$y$,
				ymin=0, ymax=100,
				xmin=0, xmax=50,
				xtick={0,20,40,60,80,100}, 
				ytick={0,22,41,83,100}
				]
				\addplot[blue, domain=0:50] {x};
			\end{axis}
		\end{tikzpicture}
	\end{subfigure}
	\begin{subfigure}[c]{0.45\linewidth}
		\begin{code}
	\begin{tikzpicture}
		\begin{axis}[
			axis lines = box,
			xlabel=$x$, ylabel=$y$,
			ymin=0, ymax=100,
			xmin=0, xmax=50,
			xtick={0,20,40,60,80,100}, 
			ytick={0,22,41,83,100}
			]
			\addplot[blue] {x};
		\end{axis}
	\end{tikzpicture}
		\end{code}
	\end{subfigure}
\end{figure}

При необходимости, можно задать и подписи к ним:
\begin{figure}[H]
	\begin{subfigure}[c]{0.55\linewidth}
		\centering
		\begin{tikzpicture}[example]
			\begin{axis}[
				axis lines = box,
				xlabel=$x$, ylabel=$y$,
				ymin=0, ymax=100,
				xmin=0, xmax=50,
				xtick={0,20,40,60,80,100}, 
				ytick={0,22,41,83,100},
				yticklabels={$y_1$, $y_2$, $y_3$, $y_4$, $y_5$},	
				]
				\addplot[blue, domain=0:50] {x};
			\end{axis}
		\end{tikzpicture}
	\end{subfigure}
	\begin{subfigure}[c]{0.45\linewidth}
		\begin{code}
	\begin{tikzpicture}
		\begin{axis}[
			axis lines = box,
			xlabel=$x$, ylabel=$y$,
			ymin=0, ymax=100,
			xmin=0, xmax=50,
			xtick={0,20,40,60,80,100}, 
			ytick={0,22,41,83,100},
			yticklabels={$y_1$, $y_2$, 
				$y_3$, $y_4$, $y_5$},
			]
			\addplot[blue] {x};
		\end{axis}
	\end{tikzpicture}
		\end{code}
	\end{subfigure}
\end{figure}
Дополнительные тики:
\begin{figure}[H]
	\begin{subfigure}[c]{0.55\linewidth}
		\centering
		\begin{tikzpicture}[example]
			\begin{axis}[
				axis lines = box,
				xlabel=$x$, ylabel=$y$,
				ymin=0, ymax=100,
				xmin=0, xmax=50,
				extra y ticks={33},	
				extra y tick labels={
					$y_{extra}$
				}, 	
				]
				\addplot[blue, domain=0:50] {x};
			\end{axis}
		\end{tikzpicture}
	\end{subfigure}
	\begin{subfigure}[c]{0.45\linewidth}
	\begin{code}
		\begin{tikzpicture}
			\begin{axis}[
				axis lines = box,
				xlabel=$x$, ylabel=$y$,
				ymin=0, ymax=100,
				xmin=0, xmax=50,
				extra y ticks={33},	
				extra y tick labels={$y_{extra}$}, 
				]
				\addplot[blue] {x};
			\end{axis}
		\end{tikzpicture}
	\end{code}
	\end{subfigure}
\end{figure}
\verb|minor tick num| задает количество промежуточных меток(minor) между основными (major):
\begin{figure}[H]
	\begin{subfigure}[c]{0.55\linewidth}
		\centering
		\begin{tikzpicture}[example]
			\begin{axis}[
				axis lines = box,
				xlabel=$x$, ylabel=$y$,
				ymin=0, ymax=100,
				xmin=0, xmax=50,
				minor tick num=2
				]
				\addplot[blue, domain=0:50] {x};
			\end{axis}
		\end{tikzpicture}
	\end{subfigure}
	\begin{subfigure}[c]{0.45\linewidth}
		\begin{code}
	\begin{tikzpicture}
		\begin{axis}[
			axis lines = box,
			xlabel=$x$, ylabel=$y$,
			ymin=0, ymax=100,
			xmin=0, xmax=50,
			minor tick num=2
			]
			\addplot[blue] {x};
		\end{axis}
	\end{tikzpicture}
		\end{code}
	\end{subfigure}
\end{figure}
Про настройки отображения чисел в PGF/TikZ более подробно показано \href{https://tikz.dev/math-numberprinting}{здесь}.
\begin{figure}[H]
	\begin{subfigure}[c]{0.55\linewidth}
		\centering
		\begin{tikzpicture}[example]
			\begin{axis}[
				axis lines = center,
				xlabel=$x$, ylabel=$y$,
				ymin=0, ymax=2,
				xmin=0, xmax=2,
				x tick label style={
					/pgf/number format/fixed zerofill}
				]
				\addplot[blue, domain=0:50] {x};
			\end{axis}
		\end{tikzpicture}
	\end{subfigure}
	\begin{subfigure}[c]{0.45\linewidth}
		\begin{code}
	\begin{tikzpicture}
		\begin{axis}[
			axis lines = center,
			xlabel=$x$, ylabel=$y$,
			ymin=0, ymax=2,
			xmin=0, xmax=2,
			x tick label style={
				/pgf/number format/fixed zerofill}
			]
			\addplot[blue] {x};
		\end{axis}
	\end{tikzpicture}
		\end{code}
	\end{subfigure}
\end{figure}
\begin{figure}[H]
	\begin{subfigure}[c]{0.55\linewidth}
		\centering
		\begin{tikzpicture}[example]
			\begin{axis}[
				axis lines = center,
				xlabel=$x$, ylabel=$y$,
				ymin=0, ymax=2,
				xmin=0, xmax=2,
				xticklabel style={
					/pgf/number format/fixed zerofill,
					 /pgf/number format/precision=4},
				yticklabel style={/pgf/number format/.cd,
					frac, frac whole=false,
					frac denom=2}
				]
				\addplot[blue, domain=0:50] {x};
			\end{axis}
		\end{tikzpicture}
	\end{subfigure}
	\begin{subfigure}[c]{0.45\linewidth}
		\begin{code}
	\begin{tikzpicture}
		\begin{axis}[
			axis lines = center,
			xlabel=$x$, ylabel=$y$,
			ymin=0, ymax=2,
			xmin=0, xmax=2,
			xticklabel style={
				/pgf/number format/fixed zerofill,
				/pgf/number format/precision=4},
			yticklabel style={
				/pgf/number format/.cd,
				frac, frac whole=false,
				frac denom=2}
			]
			\addplot[blue] {x};
		\end{axis}
	\end{tikzpicture}
		\end{code}
	\end{subfigure}
\end{figure}

\subsection{Сетка}
\begin{figure}[H]
	\begin{subfigure}[c]{0.55\linewidth}
		\centering
		\begin{tikzpicture}[example]
			\begin{axis}[
				axis lines = box,
				xlabel=$x$, ylabel=$y$,
				ymin=0, ymax=100,
				xmin=0, xmax=50,
				minor y tick num=2,
				minor x tick num=1,
				grid=major,
				grid style={}
				]
				\addplot[blue, domain=0:50] {x};
			\end{axis}
		\end{tikzpicture}
	\end{subfigure}
	\begin{subfigure}[c]{0.45\linewidth}
		\begin{code}
	\begin{tikzpicture}
		\begin{axis}[
			axis lines = box,
			xlabel=$x$, ylabel=$y$,
			ymin=0, ymax=100,
			xmin=0, xmax=50,
			minor y tick num=2,
			minor x tick num=1,
			grid=major,
			grid style={}
			]
			\addplot[blue] {x};
		\end{axis}
	\end{tikzpicture}
		\end{code}
	\end{subfigure}
\end{figure}
\begin{figure}[H]
	\begin{subfigure}[c]{0.55\linewidth}
		\centering
		\begin{tikzpicture}[example]
			\begin{axis}[
				axis lines = box,
				xlabel=$x$, ylabel=$y$,
				ymin=0, ymax=100,
				xmin=0, xmax=50,
				minor y tick num=2,
				minor x tick num=1,
				grid=minor,
				grid style={dashed}
				]
				\addplot[blue, domain=0:50] {x};
			\end{axis}
		\end{tikzpicture}
	\end{subfigure}
	\begin{subfigure}[c]{0.45\linewidth}
		\begin{code}
	\begin{tikzpicture}
		\begin{axis}[
			axis lines = box,
			xlabel=$x$, ylabel=$y$,
			ymin=0, ymax=100,
			xmin=0, xmax=50,
			minor y tick num=2,
			minor x tick num=1,
			grid=minor,
			grid style={dashed}
			]
			\addplot[blue] {x};
		\end{axis}
	\end{tikzpicture}
		\end{code}
	\end{subfigure}
\end{figure}
\begin{figure}[H]
	\begin{subfigure}[c]{0.55\linewidth}
		\centering
		\begin{tikzpicture}[example]
			\begin{axis}[
				axis lines = box,
				xlabel=$x$, ylabel=$y$,
				ymin=0, ymax=100,
				xmin=0, xmax=50,
				minor y tick num=2,
				minor x tick num=1,
				extra y ticks = {50},
				extra y tick labels = {$y_{extra}$},
				grid=both,
				major grid style={thick, red},
				minor grid style={dotted, green}
				]
				\addplot[blue, domain=0:50] {x};
			\end{axis}
		\end{tikzpicture}
	\end{subfigure}
	\begin{subfigure}[c]{0.45\linewidth}
		\begin{code}
	\begin{tikzpicture}
		\begin{axis}[
			axis lines = box,
			xlabel=$x$, ylabel=$y$,
			ymin=0, ymax=100,
			xmin=0, xmax=50,
			minor y tick num=2,
			minor x tick num=1,
			extra y ticks = {50},
			extra y tick labels = {$y_{extra}$},
			grid=both,
			major grid style={thick, red},
			minor grid style={dotted, green}
			]
			\addplot[blue] {x};
		\end{axis}
	\end{tikzpicture}
		\end{code}
	\end{subfigure}
\end{figure}
\section{Построение графика}
\begin{figure}[H]
	\begin{subfigure}[c]{0.55\linewidth}
		\centering
		\begin{tikzpicture}[example]
			\begin{axis}[
				axis lines = box,
				xlabel=$x$, ylabel=$y$,
				ymin=0, ymax=100,
				xmin=0, xmax=10,
				minor y tick num=2,
				minor x tick num=1,
				grid=major,
				grid style={}
				]
				\addplot[cyan, thick,
					domain=0:50,
					samples=300,					
					]
					{x^2};
			\end{axis}
		\end{tikzpicture}
	\end{subfigure}
	\begin{subfigure}[c]{0.45\linewidth}
		\begin{code}
	\begin{tikzpicture}
		\begin{axis}[
			axis lines = box,
			xlabel=$x$, ylabel=$y$,
			ymin=0, ymax=100,
			xmin=0, xmax=10,
			minor y tick num=2,
			minor x tick num=1,
			grid=major,
			grid style={}
			]
			\addplot[color=cyan, thick
				domain=0:50,
				samples=300
				] 
				{x^2};
		\end{axis}
	\end{tikzpicture}
		\end{code}
	\end{subfigure}
\end{figure}
\begin{figure}[H]
	\begin{subfigure}[c]{0.55\linewidth}
		\centering
		\begin{tikzpicture}[example]
			\begin{axis}[grid=both]
				\addplot coordinates {
					(0,0)
					(0.5,1)
					(0.7,3)
					(0.25,2)
					(1,2)
				};
			\end{axis}
		\end{tikzpicture}
	\end{subfigure}
	\begin{subfigure}[c]{0.45\linewidth}
		\begin{code}
	\begin{tikzpicture}
		\begin{axis}[grid=both]
			\addplot coordinates {
				(0,0)
				(0.5,1)
				(0.7,3)
				(0.25,2)
				(1,2)
			};
		\end{axis}
	\end{tikzpicture}
		\end{code}
	\end{subfigure}
\end{figure}
\begin{figure}[H]
	\begin{subfigure}[c]{0.55\linewidth}
		\centering
		\begin{tikzpicture}[example]
			\begin{axis}[grid=both]
				\addplot+[only marks] table[x index=0,y index=1] {data.dat};
			\end{axis}
		\end{tikzpicture}
	\end{subfigure}
	\begin{subfigure}[c]{0.45\linewidth}
		\begin{code}
			\begin{tikzpicture}
				\begin{axis}[grid=both]
					\addplot+[only marks] 
						table[x index=0,y index=1]
							{data.dat};
				\end{axis}
			\end{tikzpicture}
		\end{code}
	\end{subfigure}
\end{figure}
В примере выше использовано \verb|addplot+|, эта команда может автоматически подбирать параметры для графика. В приведенном примере синий цвет был выбран автоматический(\verb|addplot| отрисовала бы черные точки). Содержимое \verb|data.dat|:
\begin{Verbatim}
	20	153.2304311
	30	195.4876924
	60	212.5434827
	70	236.7057698
	80	248.2332984
	90	260.7629998
	100	267.2970842
	110	281.3195285
	10	113.9032286
	15	129.3440994
	20	170.5493006
	25	178.1146819
	35	198.7781254
	45	198.1832362
	55	205.1404073
	60	230.743486
	40	197.4683797
	50	185.2274292
\end{Verbatim}
Вместо индексов \verb|[x index=0,y index=1]| можно использовать заголовки к столбцам. Например, \verb|[x=t, y=v]|, тогда \verb|data.dat| будет таким:
\begin{Verbatim}
	t 	v
	20	153.2304311
	30	195.4876924
	...
\end{Verbatim}
Данные из \verb|.dat| файлов можно выводить сразу в табличку: 
\begin{figure}[H]
\begin{subfigure}{0.25\linewidth}
	\begin{mdframed}[backgroundcolor=yellow!20,linecolor=yellow!20]
\centering
\pgfplotstabletypeset[
column type=c,
every head row/.style={before row=\hline,after row=\hline},
every last row/.style={after row=\hline},
every last column/.style={column type/.add={}{|}},
every column/.style = {column type/.add={|}{}},
columns/x/.style={
	column name={$x$}, fixed zerofill, precision=1},
columns/y/.style={
	column name={$y(x)$}, fixed zerofill, precision=2}
]{data.dat}
\end{mdframed}
\end{subfigure}
\begin{subfigure}{0.75\linewidth}
	\centering
	\begin{code}
	\pgfplotstabletypeset[
	column type=c,
	every head row/.style={before row=\hline,after row=\hline},
	every last row/.style={after row=\hline},
	every last column/.style={column type/.add={}{|}},
	every column/.style = {column type/.add={|}{}},
	columns/x/.style={
		column name={$x$}, fixed zerofill, precision=1},
	columns/y/.style={
		column name={$y(x)$}, fixed zerofill, precision=2}
	]{data.dat}
\end{code}
\end{subfigure}
\end{figure}
Для полной настройки стиля отображения таблицы используйте \href{https://pgfplots.sourceforge.net/pgfplotstable.pdf}{документацию}.
\subsection{Погрешности на графике}
Для отображения прямоугольников ошибок надо вставить параметр \verb|error bars/.cd|
\begin{table}[H]
	\centering

	\begin{tabular}{|p{5cm}|l|p{7cm}|}
		\hline
		\verb|y dir=| или \verb|x dir=|&\verb|minus,plus,both|& задает направление ошибки: положительное, отрицательное, в обе стороны\\ \hline
		\verb|y fixed=| или \verb|x fixed=|&число&задает величину постоянной абсолютной ошибки\\ \hline
		\verb|y fixed relative=| или \verb|x fixed relative=|&число&задает величину постоянной относительной ошибки\\ \hline		
		\verb|y explicit relative| или \verb|y explicit|&–&указывает на необходимость считать ошибки из файла. В параметре надо указать столбец с ошибками \verb|table[y error index=2]|\\ \hline
	\end{tabular}
\end{table}
\begin{figure}[H]
	\begin{subfigure}[c]{0.55\linewidth}
		\centering
		\begin{tikzpicture}[example]
			\begin{axis}[grid=both]
				\addplot+[only marks]
				plot[
				error bars/.cd,
				y dir=both,
				y fixed relative=0.1, 
				x dir = minus,
				x fixed = 0.4
				]
				table[x index=0,y index=1] {data1.dat};
			\end{axis}
		\end{tikzpicture}
	\end{subfigure}
	\begin{subfigure}[c]{0.45\linewidth}
		\begin{code}
	\begin{tikzpicture}
		\begin{axis}[grid=both]
			\addplot+[only marks]
			plot[
			error bars/.cd,
			y dir=both,
			y fixed relative=0.1,
			x dir = minus,
			x fixed = 0.4
			]
			table[x index=0,y index=1] 
			{data.dat};
		\end{axis}
	\end{tikzpicture}
		\end{code}
	\end{subfigure}
\end{figure}
\subsection{Линия тренда}
\begin{figure}[H]
	\begin{subfigure}[c]{0.55\linewidth}
		\centering
		\begin{tikzpicture}[example]
			\begin{axis}[grid=both]
				\addplot[only marks]
				table[x index=0,y index=1] {data1.dat};
				
				\addplot[red, thin] table [
				x =x,
				y={create col/linear regression={y=y}}]
				{data1.dat};
				
			\end{axis}

		\end{tikzpicture}
	\end{subfigure}
	\begin{subfigure}[c]{0.45\linewidth}
		\begin{code}
	\begin{tikzpicture}
		\begin{axis}[grid=both]
			\addplot[only marks]
			table[x index=0,y index=1]
			{data.dat};
			
			\addplot[red, thin] table [
			x=x,
			y={create col/linear regression=
				{y=y}}]
			{data.dat};
		\end{axis}
	\end{tikzpicture}
		\end{code}
	\end{subfigure}
\end{figure}
Также можно получить коэффициенты фитирующей прямой следующими командами:\\ \verb|\pgfplotstableregressiona| и \verb|\pgfplotstableregressionb|\\
Для автоматической записи уравнения фитирующей прямой в легенду графика можно использовать команду: 
\[\verb|\addlegendentry{$\pgfmathprintnumber{\pgfplotstableregressiona} \cdot x$ + |\]
\[\verb|+ $\pgfmathprintnumber{\pgfplotstableregressionb}$}|\]

\subsection{Легенда графика}
\begin{figure}[H]
	\begin{subfigure}[c]{0.55\linewidth}
		\centering
		\begin{tikzpicture}[example]
			\begin{axis}[grid=both, legend style={fill = yellow!20}]
				\addplot[only marks , mark = triangle*, mark options = {color=red,scale=1}]
				table[x=H, y=m] {data2.dat};
%				\addplot[red] table [x=H, y={create col/linear regression={y=m}}] {data2.dat};
				
				\addplot[only marks , mark = triangle*, mark options = {color=red!60,scale=1}]
				table[x=H, y=m] {data3.dat};
%				\addplot[red!60] table [x=H, y={create col/linear regression={y=m}}] {data3.dat};
				
				\legend{exp1, exp2}
			\end{axis}
			
		\end{tikzpicture}
	\end{subfigure}
	\begin{subfigure}[c]{0.45\linewidth}
		\begin{code}
	\begin{tikzpicture}
		\begin{axis}[grid=both, legend style={fill = yellow!20}]
			\addplot[only marks,
				mark = triangle*,
				mark options = {color=red}]
			table[x=X, y=Y] {data2.dat};
			\addplot[only marks,
				mark = triangle*,
				mark options={color=red!60}]
			table[x=X, y=Y] {data3.dat};
			\legend{exp1, exp2}
		\end{axis}
	\end{tikzpicture}
		\end{code}
	\end{subfigure}
\end{figure}
\begin{figure}[H]
	\begin{subfigure}[c]{0.55\linewidth}
		\centering
		\begin{tikzpicture}[example]
			\begin{axis}[grid=both,
					legend style = {
						fill=pink,
						cells={anchor=east},
						legend pos=outer north east,}]
				\addplot[only marks , mark = triangle*, mark options = {color=red,scale=1}]
				table[x=H, y=m] {data2.dat};
				%				\addplot[red] table [x=H, y={create col/linear regression={y=m}}] {data2.dat};
				
				\addplot[only marks , mark = triangle*, mark options = {color=red!60,scale=1}]
				table[x=H, y=m] {data3.dat};
				%				\addplot[red!60] table [x=H, y={create col/linear regression={y=m}}] {data3.dat};
				
				\legend{exp1, exp2}
			\end{axis}
		\end{tikzpicture}
	\end{subfigure}
	\begin{subfigure}[c]{0.45\linewidth}
		\begin{code}
	\begin{tikzpicture}
		\begin{axis}[grid=both,
			legend style={
				fill=pink,
				cells={anchor=east},
				legend pos=outer north east,}]
		...
		\end{axis}
	\end{tikzpicture}
		\end{code}
	\end{subfigure}
\end{figure}
\subsection{Пины}
\begin{figure}[H]
	\begin{subfigure}[c]{0.55\linewidth}
		\centering
		\begin{tikzpicture}[example]
			\begin{axis}[grid=both]
				\addplot[only marks , mark = triangle*, mark options = {color=red}]
				table[x=x,  y=y] {data4.dat}
				coordinate [pos=0.25] (A)
				coordinate [pos=0.4] (B)
				;
				
				\draw (A) -| (B) node [pos=0.75,anchor=west]
				{$k$};
				\node [pin=200:Выброс] at (16, 3) {};

			\end{axis}
			
		\end{tikzpicture}
	\end{subfigure}
	\begin{subfigure}[c]{0.45\linewidth}
		\begin{code}
	\begin{tikzpicture}
		\begin{axis}[grid=both]
			\addplot[only marks,
				mark = triangle*,
				mark options={color=red}]
			table[x=x,  y=y] {data4.dat}
			coordinate [pos=0.25] (A)
			coordinate [pos=0.4] (B)
			;
			\draw (A) -| (B) node
				[pos=0.75,anchor=west] {$k$};
			\node [pin=200:Выброс] 
				at (16, 3) {};
		\end{axis}
	\end{tikzpicture}
\end{code}
	\end{subfigure}
\end{figure}
\end{document}
